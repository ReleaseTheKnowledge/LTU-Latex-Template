%%%%%%%%%%%%%%% Hur man använder denna mall %%%%%%%%%%%%%%%

%%%     Titelsida
Ändra variablerna i "Essential/Title.tex".
% Hur titelsidan ser ut genereras längre ner i samma kod

%%%     Sections
Skriv direkt i filerna "Sections/...tex".
Radera de filer du inte vill ha. Så löser koden resten.
% Filen main.tex skapar sections och lägger ihop allt. 
% Vill du ändra sections eller lägga till andra filer så ändra där.

%%%     Lägg till bilagor
Lägg in bilagor i "Sections/Bilagor/".
Namnge dom $ "Bilaga_1", "Bilaga_2",.. $ Så läggs dom automatiskt in i dokumentet.
Detta funkar för .tex och .pdf.
% koden för \bilagaloop är skriven i "Essential/Packages.tex"

%%%     Namnge bilagor
Namnge bilagor i "Essential/Packages.tex".
Lägg in namn i \edef\bilaganamn{{"namn 1","namn 2",""}}. 
% Första namnet ges till bilaga 1, nästa namn till bilaga 2,..
% Det blir "Bilaga 1, namn 1",.. 
% Lägg till så många som behövs.

%%%     Referenser
Lägg till referenser i "Essential/references.bib"


%%% SNABB COMPILE!, --> "Essential/Packages.tex". Bra för många bilder!
%%% Ändra språk,    --> "Essential/Packages.tex".
%%% Fler packages,  --> "Essential/Packages.tex". Under % Extra
%%% För många namn?,--> "Essential/Titel.tex", rad 38 och 45/48.


%%%%%%%%%%%%%%%%%%%%%%%%%%%%%%%%%%%%%%%%%%%%%%%%%%%%%%%%%%%%%
%   Nice stuff

Hänvisning till \cref{eq:namn}
\begin{equation} \label{eq:namn}
\end{equation}


Hänvisning till \cref{fig:namn}.
\begin{figure} [H]
    \centering 
    \includegraphics[width=0.5\textheight]{bild.png}
    \caption{Beskrivande text.}
    \label{fig:namn}
\end{figure}


Hänvisning till \cref{tab:namn}.
\begin{table}[H]
\caption{Namn} \label{tab:namn}.
\centering 
    \begin{tabular}{|c|c|c|}
    \hline
    \textbf{} & \textbf{} & \textbf{} \\
     \hline
 &  &  \\
    \hline
 &  &  \\
    \hline
 &  &  \\
    \hline
    \end{tabular} 
\end{table}