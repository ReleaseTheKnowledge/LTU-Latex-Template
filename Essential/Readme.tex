%-----------------------------------------------
%%        Hur man använder denna mall!
%-----------------------------------------------

%   Titelsida
Ändra variablerna i "Essential/Title.tex".
% Hur titelsidan ser ut genereras längre ner i samma kod.

%   Sections
Skriv direkt i filerna "Sections/...tex".
Radera de filer du inte vill ha. Så löser koden resten.
% Filen main.tex skapar sections och lägger ihop allt. 
% Vill du ändra rubriker eller lägga till andra filer så ändra i main.tex.

%   Lägg till bilagor
Lägg in bilagor i "Sections/Bilagor/".
Namnge dom $"Bilaga_1", "Bilaga_2",..$ så läggs dom automatiskt in i dokumentet.
Detta funkar för .tex filer.
% koden för \bilagaloop är skriven i "Essential/Packages.tex".

%   Namnge bilagor
Namnge bilagor höst upp i "Sections/Main.tex".
Lägg in namn i \edef\bilaganamn{{"namn 1","namn 2",""}}. 
% Första namnet ges till bilaga 1, nästa namn till bilaga 2,..
% Det blir "Bilaga 1, namn 1",.. 
% Lägg till så många som behövs.

%   Referenser
Lägg till referenser i "Essential/References.bib".

%   Newpage
Vill du ha ny sida för varje section? Ta bort procent framför "\newpage" i "main.tex".

%%% SNABB COMPILE!, --> "Essential/Packages.tex". Bra för många bilder!
%%% Ändra språk,    --> "Essential/Packages.tex".
%%% Fler packages,  --> "Essential/Packages.tex". Nedanför %   Extra.
%%% Mer plats på titelsida,--> "Essential/Titel.tex", rad 38 och 45/48.

%-----------------------------------------------
%%                 Nice stuff
%-----------------------------------------------

Hänvisning till \cref{eq:namn}
\begin{equation} \label{eq:namn}
\end{equation}


Hänvisning till \cref{fig:namn}.
\begin{figure} [H]
    \centering 
    \includegraphics[width=0.5\textheight]{bild.png}
    \caption{Beskrivande text.}
    \label{fig:namn}
\end{figure}


Hänvisning till \cref{tab:namn}.
\begin{table}[H]
\centering
\caption{Namn} \label{tab:namn}
\begin{tabular}{ l l l } \toprule
\textbf{} & \textbf{} & \textbf{} \\
\midrule
     &  &  \\
     &  &  \\
     &  &  \\
\bottomrule
\end{tabular}
\end{table}


%-----------------------------------------------
%%                    Länkar
%-----------------------------------------------

% Mattematiska symboler
% http://tug.ctan.org/info/undergradmath/undergradmath.pdf

% Draw symbols
% http://detexify.kirelabs.org/classify.html 

% ALLA symboler
% http://tug.ctan.org/info/symbols/comprehensive/symbols-a4.pdf

% Generera tabeller
% http://www.tablesgenerator.com/ 

% Hur man gör bra tabeller
% https://inf.ethz.ch/personal/markusp/teaching/guides/guide-tables.pdf

% Skriva si enheter / nummer
% http://ctan.math.utah.edu/ctan/tex-archive/macros/latex/contrib/siunitx/siunitx.pdf

